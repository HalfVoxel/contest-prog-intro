När du skriver ett program i C++ skriver du läsbar text. När datorn sedan ska köra din kod måste denna först göras om till något datorn kan läsa. Detta är \emph{kompilatorns} jobb.

Då det är väldigt jobbigt att få C++ att fungera på ett normalt sätt i Windows rekommenderar och antar vi att du kör någon Linux-variant. Exempelvis är Ubuntu (\texttt{http://www.ubuntu.com/}) lätt att installera. Det kan till och med installeras direkt från Windows m h a något som kallas Wubi. Andra alternativ inkluderar Cygwin (\texttt{http://www.cygwin.com/}) och MSYS (\texttt{http://www.mingw.org/wiki/msys/}). 

\texttt{g++} (en del av GNU Compiler Collection) är den kompilator som oftast används inom tävlingssammanhang, och den som antas när kod kompileras i detta häfte - mer specfikt version 4.6.3 och senare.

För att kompilera en fil använder du kommandot

\texttt{\$ g++ -Wall -O2 <filnamn>}

i en terminal eller kommandoprompt - det som i Windows kallas \texttt{cmd.exe}. En fil med namnet \texttt{a.out} eller \texttt{a.exe} kommer då att skapas - detta är filen som du ska köra med hjälp av

\texttt{\$ ./a.out}
