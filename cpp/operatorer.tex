Nu har vi våra datatyper och variabler. Så, hur gör vi någonting vettigt med dem? Vi har redan sett exempel på två \texttt{operatorer} - additionsoperatorn \texttt{+} och tilldelningsoperatorn \texttt{=}.

Det finns ett antal aritmetiska operatorer i C++ som är väldigt bra att kunna:

\lstinputlisting{cpp/operatorer.cpp}

Vid division med noll (antingen genom / eller \%), kommer ett fel uppstå och ditt program avslutas.

Utöver dessa har vi en mängd operatorer som används för att skapa booleska uttryck:

\lstinputlisting{cpp/operatorer2.cpp}

Prova att leka med operatorerna och se hur de beter sig, speciellt för olika datatyper och kombinationer (hur fungerar division mellan ett heltal och ett flyttal?).
