Datorns minne är egentligen bara en lång lista med ettor och nollor. Eftersom det dock är svårt att använda minnet i denna form låter C++ och operativsystemet oss att göra saker lite lättare. Det främsta exemplet är att vi istället för att referera till var i listan någonting finns, kan använda oss utav ett namn med bokstäver och låta datorn klura ut vilken plats i minnet vi menar - en variabel.

Variabler liknar matematikens variabler, med vissa undantag. Främst behöver vi i C++ veta vad för sorts information vi sparar i en variabel. Är det ett heltal, ett decimaltal, en textsträng, eller något helt annat? Antag exempelvis att vi vill säga åt datorn att variablen $x$ är lika med talet 5.

Detta gör vi med satsen:
\begin{lstlisting}
int x = 5;
\end{lstlisting}

Vi säger här två saker till datorn; först och främst att vi vill ha en variabel $x$ som är ett heltal (int står för engelskans integer, som betyder just heltal) och sedan att $x$ just nu ska ha värdet 5. När vi väl sagt vilken typ $x$ är varken behöver eller får vi säga det igen. En variabel kan alltså aldrig ändra typ efter att den deklarerats.

Till skillnad från matematiska variabler kan vi dock ändra C++-variabler lite hur vi vill. Kanske ändrar vi oss och vill att $x$ ska vara lika med 42 istället. Då gör vi på föjande vis:

\begin{lstlisting}
int x = 5; //x är nu lika med 5
x = 42; //Nu är x lika med 42! Observera avsaknaden av "int" i början
\end{lstlisting}

Kanske vill vi till och med skapa en ny variabel $youJustLostTheGame$ som ska vara lika med $x+2$.

Såhär skulle det se ut:

\begin{lstlisting}
int x = 5;
x = 42;
int youJustLostTheGame = x+2; // = 44
//Om vi nu ändrar x, är youJustLostTheGame fortfarande
//44 - det ändras inte tillsammans med x.
x = 5;
\end{lstlisting}

Det är viktigt här att skilja på hur matematiska variabler skiljer sig från C++. Trots att vi precis sagt att $youJustLostTheGame$ är lika med $x+2$, kommer variabeln \emph{inte} ändras när vi i efterföljande sats sätter $x = 5$. Tilldelningsmodellen skiljer sig alltså från hur algebra fungerar, vilket ofta är förvirrande för nya programmerare.

Variabelnamn får innehålla bokstäver, siffror eller understreck, förutsatt att det inte börjar på en siffra samt inte är någon av de nyckelord som finns i C++.

Det finns många fler exempel på de typer en variabel kan ha. Här visar vi exmpel på några deklarationer:

\begin{lstlisting}
//En char kan lagra ett tecken. Tecken skrivs omgivna av ''.
char ettA = 'a';

//En int kan bara spara tal i intervallet -2147483648 ( = -2^31)
//till 2147483647 ( = 2^31-1)
//En long long kan däremot spara tal mellan 
//ungefär -10^18 (-2^63) till 10^18 (2^63-1)
//Observera prefixet "LL ( = long long)" på stora tal
long long stortTal = 10000000000000LL;

//Värden som bara kan vara sanna eller falska är också vanliga.
//Dessa kallas för booleans och kan ha antingen värdet true eller false.
//Dock fuskar C++ och betraktar allt som inte är 0 som true
//och 0 som false.
bool isJohanAwesome = true;

//Tal med decimaler sparas i doubles. Dessa kan hålla tal
//i intervallet +/- 1.7^(+/- 308). Dock är de inte exakta, utan bara
//en uppskattning (c:a 16 siffrors noggrannhet)
double speedOfAnUnladenSwallow = 15.6132;

//Text-strängar sparar C++ med typen "string" som är en del
//av det inbyggda standardbiblioteket
string herp = "derp";
\end{lstlisting}
