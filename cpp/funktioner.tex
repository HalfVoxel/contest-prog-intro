Vi har redan sett exempel på en funktion - \emph{main} - och ska nu gå in djupare på vad de är och varför de är användbara.

Under avsnittet om villkorsuttryck såg vi hur man kunde beräkna absolutbeloppet av ett heltal med kod. Dock är det en aning omständligt att behöva skriva en if-sats varje gång man vill beräkna beloppet, och det är inte heller särskilt läsbart. Istället skulle man gärna vilja ha ett sätt att ta den operationen och ge den ett symboliskt namn, t ex $abs$. För att utföra operationen på ett tal $x$ skulle man sedan skriva $abs(x)$. Detta är precis vad funktioner gör:


\lstinputlisting{cpp/funktioner.cpp}

För att kunna använda en funktion i C++ så måste den vara definerad tidigare i koden än den används. Det är bland annat därför man har \#include kommandona högst upp i koden. Eftersom det de är till stor del att definera funktioner så måste de vara definerade innan resten av koden.

Funktioner behöver inte returnera något heller. Det är ofta man vill ha en funktion som endast gör något men inte nödvändigtvis ger tillbaka ett svar. För att illustrera detta så behöver vi också nämna att variabler, som vi hittils endast har deklarerat innuti funktioner också kan vara globala. Dvs de är inte innuti någon funktion utan är tillgängliga från alla funktioner. Då en funktion inte ska returnera något så skriver man returntypen ``void" (iställer för ``int" som vi har använt hittils)

\lstinputlisting{cpp/funktioner2.cpp}

Funktioner kan också ta flera argument, eller inga argument alls (som main funktionen). Vi kan till exempel göra en funktion som räknar ut arean av en triangel givet sidlängderna på den (Herons formel ftw! Höjd och bredd är tråkigt).

\lstinputlisting{cpp/funktioner3.cpp}

Funktioner kan som vi har sett anropa andra funktioner, men de kan också andropa sig själva vilket kan vara väldigt användbart.
Här är en enkel rekursiv funktion.
$$
f(x) = 
\left\{
    \begin{array}{ll}
        x & \mbox{om } x \leq 5 \\
        f (x/2) & \mbox{om } x \mbox{ är jämnt} \\
        f ((x-1)*2) & \mbox{annars}
    \end{array}
\right.
$$

Vi ska skriva en funktion som fungerar som ovan och skriver ut varje steg.

\lstinputlisting{cpp/funktioner4.cpp}

I alla ovanstående funktioner har vi skickat argument, men om vi skickar variabeln x så kommer allt som funktionen gör med den inte att påverka orginal variabeln.
Skriver vi ett \& tecken framför argumentnamnet i funktionsdeklarationen så kommer funktionen att skickas som en referens och inte som ett värde.

\lstinputlisting{cpp/funktioner5.cpp}

Om du kör programmet så ser du att den första funktionen inte kommer att ändra på orginalvariablen, medans den andra funktionen kommer att göra det.