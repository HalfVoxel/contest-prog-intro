Vi har redan sett exempel på en funktion - \emph{main} - och ska nu gå in djupare på vad de är och varför de är användbara.

Under avsnittet om villkorsuttryck såg vi hur man kunde beräkna absolutbeloppet av ett heltal med kod. Dock är det en aning omständligt att behöva skriva en if-sats varje gång man vill beräkna beloppet, och det är inte heller särskilt läsbart. Istället skulle man gärna vilja ha ett sätt att ta den operationen och ge den ett symboliskt namn, t ex $abs$. För att utföra operationen på ett tal $x$ skulle man sedan skriva $abs(x)$. Detta är precis vad funktioner gör:


\lstinputlisting{cpp/funktioner.cpp}

För att kunna använda en funktion i C++ så måste den vara definerad tidigare i koden än den används. Det är bland annat därför man har \#include kommandona högst upp i koden. Eftersom det de är till stor del att definera funktioner så måste de vara definerade innan resten av koden.

Funktioner behöver inte returnera något heller. Det är ofta man vill ha en funktion som endast gör något men inte nödvändigtvis ger tillbaka ett svar. För att illustrera detta så behöver vi också nämna att variabler, som vi hittils endast har deklarerat innuti funktioner också kan vara globala. Dvs de är inte innuti någon funktion utan är tillgängliga från alla funktioner. Då en funktion inte ska returnera något så skriver man returntypen ``void" (iställer för ``int" som vi har använt hittils)

\begin{lstlisting}
#include <iostream>

using namespace std;

int counter;
int addToCounter (int num) {
	counter += num;
	cout << "Summan av alla tal är nu: " << counter << endl;
}

int main () {
	while (true) {
		int tal;
		cout << "Skriv ett tal: "<< endl;
		cin >> tal;
		addToCounter (tal);
	}
}
\end{lstlisting}