\documentclass[10pt,a4paper]{article}

% Package declarations

\usepackage[utf8]{inputenc}
\usepackage[swedish]{babel}
\usepackage{amsmath}
\usepackage{amsfonts}
\usepackage{amssymb}
\usepackage{graphicx}
\usepackage[parfill]{parskip}
\usepackage{amsthm}
\usepackage[bookmarks]{hyperref}
\usepackage{pstricks}

\begingroup
    \makeatletter
    \@for\theoremstyle:=definition,remark,plain\do{%
        \expandafter\g@addto@macro\csname th@\theoremstyle\endcsname{%
            \addtolength\thm@preskip\parskip
            }%
        }
\endgroup

\newtheoremstyle{problem}
  {3pt}% space before
  {3pt}% space after
  {}% body font
  {}% indent
  {\bfseries}% header font
  {.}% punctuation
  {.5em}% after theorem header
  {}% header specification (empty for default)
\makeatother

\newtheorem{defn}{Definition}
\newtheorem{sats}{Sats}

\theoremstyle{problem}
\newtheorem{problem}{Problem}
\newtheorem{exercise}{Övning}
\newtheorem{example}{Exempel}

% Begin title, ToC

\author{Johan Sannemo}
\title{Grafteori}

\begin{document}

\Large{ALGORITMISK GRAFTEORI}
\\
\small{Introduktion till Tävlingsprogrammering}

\section{Grunder}

Grafteori är ett av de vanligast förekommande områdena i BOI och IOI. Det är studien av objekt som kallas \emph{grafer}, och har stor betydelse
inom allt från nätverk till trafikplanering.

Säg till exempel att vi har 6 stycken korsningar, numrerade från 1 - 6. Vi vet att det finns en väg mellan korsrningarna $(1,2), (2,3), (3, 4), (2, 4)$ och $(5, 5)$.

Visuellt kan vi representera detta med följande bild:

\begin{figure}[htp]
    \label{fig:exempelGrafer}
    \centering{\includegraphics[scale=0.5]{oriktad_graf.pdf}}
\end{figure}

Om vägarna istället är enkelriktade kan vi istället rita en pil mellan korsningarna:

\begin{figure}[htp]
    \label{fig:exempelGrafer}
    \centering{\includegraphics[scale=0.5]{riktad_graf.pdf}}
\end{figure}

Studien av denna sorts objekt kallas för \emph{grafteori}.

\begin{defn}
    En \emph{graf} $G$ består av två mängder $(V, E)$. Mängden $V$ kallas för grafens \emph{noder}, och är sammanbundna av sk \emph{kanter} där varje kant binder samman två noder.
\end{defn}

En kant $(x, y)$ skrivs ibland kort som $xy$.

Vårt lilla vägnätverk ovan kan vi alltså representera som en graf - mängden $V$ är då de olika korsningarna, och mängden $E$ består av de olika vägarna.

Skillnaden mellan våra två vägnätverk ovan är att kanterna har en viss riktning

Skillnaden mellan den riktade och oriktade grafen är att kanterna är orienterade; de ''pekar'' från en av noderna till den andra.
Ritar vi en graf spelar det inte någon roll hur vi ritar noderna och kanterna så länge varje kant ansluter rätt par av noder.

Oftast har noder inte kanter till sig själv (sk loopar) som i vårt exempel med $(5, 5)$, och det finns inte två kanter mellan samma par av noder (sk parallella kanter), om de inte är riktade åt olika
håll i en riktad graf. Om en uppgift inte specifierar huruvida loopar eller parallella kan förekomma eller inte är det säkrast att anta att de kan finnas.

\begin{defn}
Två noder $x, y$ är grannar i grafen $G = (V, E)$ om kanten $(x, y)$ finns i $E$.
\end{defn}

I exemplet ovan har t.ex. noden 3 grannarna 2 och 4.

\begin{defn}
Graden $\deg(v)$ för en nod $v$ är antalet kanter med någon av ändpunkterna i $v$, där loopar räknas två gånger. I en riktad graf används termen \emph{utgrad} för antalet kanter som börjar i $v$ och \emph{ingrad}
för kanter som slutar i $v$. Om $\deg(v) = 0$ säger vi att $v$ är \emph{isolerad}.
\end{defn}


Vi har nu gått igenom tillräckligt med teori för att bevisa vår första sats:

\begin{sats}
\emph{(Handskakningslemmat)}
    Det finns ett jämnt antal noder med udda grad i en graf.
\begin{proof}
    Summan av alla grader räknar antalet kanter två gånger, dvs $\sum_{v \in V}{\deg(v)} = 2|E|$.
    Eftersom HL är ett jämnt tal måste också VL vara det. Summan av ett udda antal udda tal är udda,
    så VL är endast jämnt om antalet udda grader är jämnt.
\end{proof}
\end{sats}


\begin{example}
I ett örike finns ett antal tvåvägs-flygförbindelser mellan öarna. Varje ö har 2, 4 eller 6 flygförbindelser, förutom Pauls Ö som har 3 förbindelser,
och Huvudön som har 7 förbindelser. Visa att man, med mellanlandningar, kan flyga från Pauls Ö till Huvudön. (av Paul Vaderlind)
\begin{proof}
Vi börjar med att bilda grafen $G$ med öarna som noder och flygförbindelser som kanter. Vi delar upp $G$ i två grafer $G_p$ och $G_h$. Dessa består av alla
öar och flygförbindelser man kan nå från Pauls ö respektive Huvudön. Antag $G_p \not= G_h$ - dvs, vi kan inte nå Pauls Ö från Huvudön. Då har $G_p$ en nod med udda grad (Pauls Ö), och resten har jämn grad
(2, 4 eller 6). Detta är en motsägelse enligt handskakningslemmat, så $G_p = G_h$ och Pauls Ö kan nå Huvudön.
\end{proof}
\end{example}


\begin{defn}
    En graf $G = (V, E)$ kallas \emph{viktad} om vi till varje kant $e$ i mängden $E$ ordnar ett tal $w(e)$. Vi kallar detta kantens \emph{vikt}.
    Vi definerar $w(e) = \infty$ då $e$ inte är i $E$. I en oviktad graf låter vi $w(e) = 1$ för alla $e$.
\end{defn}

En viktad graf kan t ex representera vägarnas längd i det tidigare exemplet om en stads vägnät.

\begin{defn}
    En \emph{väg} $W = (V, E)$ är en sekvens av noder $V = (v_0, v_1, ..., v_n)$. Väg \emph{längd} är summan av dessa kanters vikter, dvs $\sum_{i = 0}^{n-1}{w(v_iv_{i+1})}$.
    Om varje $v_i$ är olika kallar vi vägen för \emph{enkel}.
    Om $v_0 = v_n$ är $W$ en \emph{cykel}.
\end{defn}

\begin{defn}
	Om en graf saknar cyckler kallar vi den för \emph{acyklisk}.
\end{defn}

En cykel är alltså en väg, där man lagt till kanten mellan den sista och den första noden. I vår exempelgraf är $(2, 3, 4, 2)$ en cykel - både
i den riktade och den oriktade grafen. I den riktade grafen är däremot inte $(2, 4, 3, 2)$ en cykel. Det är den däremot i den oriktade.

\begin{defn}
    En oriktad graf $G = (V, E)$ är \emph{sammanhängande} om det för varje par $v, u \in V$ existerar en väg mellan dem.
    En riktad graf $G = (V, E)$ är \emph{starkt sammanhängande} om det för varje par $v, u \in V$ existerar en väg mellan dem.
\end{defn}

Vår graf ovan är inte en sammanhängande graf, ty det inte finns en väg från nod 5 till 6. Däremot är den oriktade grafen $V = \{1, 2, 3, 4\}$ sammanhängande,
Den riktade grafen $V = \{2, 3, 4\}$ är också starkt sammanhängande, eftersom vi kan ta oss mellan varje par av noderna 2, 3, och 4.

\subsection{Övningar}
\begin{exercise}
Visa att i en oriktad, sammanhängande graf $G = (V, E)$ gäller $|E| \ge |V| - 1$.
\end{exercise}
\begin{exercise}
Visa att alla noder i en oriktad graf inte kan ha olika grader.
\end{exercise}
\begin{exercise}
Givet en oriktad graf $G = (V, E)$, visa att om $|E| \ge |V|$ har $G$ en cykel.
\end{exercise}
\begin{exercise}
Visa att det i varje riktad, acyklisk graf existerar en nod med ingrad 0.
\end{exercise}
\begin{exercise}
I en oriktad, sammanhängande graf $G = (V, E)$ gäller $\deg(v) \ge k \ge 2$ för alla $v \in V$. Visa att det finns en väg med längd minst $k$.
\end{exercise}

\subsection{Problem}
\begin{problem}
Hitta en $O(|V| + |E|)$-algoritm för att avgöra om en graf är sammanhängande eller inte (problemet \texttt{connectivity} på KATTIS).
\end{problem}
\begin{problem}
I en oriktad graf $G$ är $V(G) = \{v_1, ..., v_n\}$. Givet graderna $\deg(v_i)$ för alla $1 \le i < n$,
beräkna alla värden som $\deg(v_n)$ kan anta. (problemet \texttt{degrees} på KATTIS).
\end{problem}

\section{Representation av grafer}
För att representera en graf i datorn är ju en bild synnerligen opassande. De vanligaste sätten att spara grafer är istället \emph{grannmatriser} och \emph{grannlistor}.
Ibland måste man också hantera \emph{implicita} grafer; fall där grafen inte är explicit given, t ex ställningarna i ett spel

\subsection{Grannmatriser}
\begin{defn}
Om $G = (V, E)$ är en riktad, viktad graf och $V = \{v_1, ..., v_n\}$ är grafens \emph{grannmatris}  den $|V| \times |V|$-matris $A = (a_{i,j})$, där $a_{i,j} = w(v_i v_j)$.
\end{defn}

\begin{defn}
Om $G = (V, E)$ är en riktad men oviktad graf låter vi istället $a_{i, j}$ vara antalet kanter från $v_i$ till $v_j$. Här förutsätts alltså inte
att grafen är enkel, utan vi kan ha godtyckligt många kanter i vardera riktning mellan två noder.
\end{defn}

Denna representation använder $O(|V|^2)$ minne, och tar $O(1)$ tid för att lägga till, modifiera och ta bort kanter om den implementeras
som en tvådimensionell vektor. Grannmatriser klarar inte av att hantera multipla kanter, förutom i fallet med oviktade grafer. Där har istället matrisen $A$ en
hel del användbara kombinatoriska egenskaper; matrisen $A^n$ representerar då antalet stigar av längd $n$ mellan två noder. Detta kommer vi använda oss av senare.

Grannmatriser är bäst att använda när $|V|^2 \approx |E|$, d v s när grafen är tät.

Vi ger grannmatrisen för den riktade, oviktade exempelgrafen i figur \ref{fig:exempelGrafer}:

\[ \left( \begin{array}{cccccc}
0 & 1 & 0 & 0 & 0 & 0 \\
0 & 0 & 1 & 0 & 0 & 0 \\
0 & 0 & 0 & 1 & 0 & 0 \\
0 & 1 & 0 & 0 & 0 & 0 \\
0 & 0 & 0 & 0 & 1 & 0 \\
0 & 0 & 0 & 0 & 0 & 0
\end{array} \right)\]

\subsection{Grannlistor}
Ett annat sätt att representera grafer är som listor över alla kanter som en nod har. Detta kräver bara $O(|E|)$ minne, vilket är bättre när man har
relativt få kanter i grafen. Om man för detta använder en \texttt{vector} får man $O(1)$ tilläggning och borttagning av kanter, men det tar $O(|V|)$ för att
avgöra om en viss kant existerar. I fall där man bara konstruerar grafen en gång och sedan itererar genom kanterna för en viss nod är alltså
tidskomplexiteten samma som för grannmatrisen; men med markant bättre tids- och minneskomplexitet när grafen är gles.

\subsection{Grannmaps}
En \emph{grannmap} kombinerar grannmatrisen och grannlistan för att få det bästa från båda världar. Istället för att varje nod har en lista av kanter använder
man en map (\texttt{map} i C++) som mappar mål-nod till kantvikten. I en oviktad graf utan vikter kan man istället använda ett set (\texttt{set} i C++) med alla grannar.

Den använder $O(|E|)$ minne, och har komplexiteten $O(\log |V|)$ för tilläggning,
borttagning och uppslagning. Den är alltså långsammare än grannlistan när man ska iterera genom kanter, men om man blandar uppslagning och iterering tjänar
man snabbt på att använda en grannmap.

\subsection{Övningar}
\begin{exercise}
Givet en graf $G = (V, E)$, vilken/vilka representationer är lämpliga om:
\begin{description}
\item[a)]{$|V| = 1000$ och $|E| = 499500$}
\item[b)]{$|V| = 10000$ och $|E| = 20000$}
\item[c)]{$|V| = 1000$ och $|E| = 10000$}
\end{description}
\end{exercise}
\begin{exercise}
Bevisa egenskapen hos grannmatrisen $A$ för oviktade grafer, alltså att $A^n$ ger antal stigar av längd $n$ mellan två noder.
\end{exercise}
\subsection{Problem}
\begin{problem}
Skriv ett program som läser in en graf, sparar den i lämpligt format och svarar på frågor om vilka grannar olika noder har (problemet \texttt{neighbours} i KATTIS).
\end{problem}

\section{Sökning i grafer}
\subsection{DFS}
\subsection{BFS}

\end{document}
