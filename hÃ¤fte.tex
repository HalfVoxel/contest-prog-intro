\documentclass[10pt,a4paper]{report}

% Package declarations

\usepackage[utf8]{inputenc}
\usepackage[swedish]{babel}
\usepackage{amsmath}
\usepackage{amsfonts}
\usepackage{amssymb}
\usepackage{graphicx}
\usepackage{parskip}
\usepackage[left=2cm,right=2cm,top=2cm,bottom=2cm]{geometry}
\usepackage[bookmarks]{hyperref}
\usepackage{listings}

% End package declarations

% Listings configuration
\lstset{%
language=C++,
showstringspaces=false,
numbers=left,
tabsize=2,
}

\lstset{breaklines=true, breakatwhitespace=true}

\lstset{literate=%
{Ö}{{\"O}}1
{Ä}{{\"A}}1
{Ü}{{\"U}}1
{ß}{{\ss}}2
{ü}{{\"u}}1
{ä}{{\"a}}1
{ö}{{\"o}}1
{å}{{\aa}}1
{Å}{{\AA}}2
}

% End listings configuration

% Begin title, ToC

\author{Johan Sannemo}
\title{Introduktion till Tävlingsprogrammering}
\date{Augusti 2012}

\begin{document}
\maketitle
\tableofcontents

% End title, ToC

% Begin content

\chapter{En snabbkurs i C++}

\section{Grunder}
C++ är det överlägset vanligaste och populäraste programmeringsspråket som används inom tävlingsprogrammering. Dels är det väldigt snabbt och dels är det ett väldigt kraftfullt språk med många inbyggda funktioner som lämpar sig för tävlingsprogrammering.

Denna snabbkurs kommer gå igenom långt ifrån hela språket, utan fokuserar istället på vad som är nödvändigt för att kunna använda det i tävlingar. Innehållet är väldigt komprimerat och använder sig mycket av kodexempel istället för text. Det rekommenderas därför att du har en dator till hands och kan prova allting du läser själv för att få ut så mycket som möjligt.	
\subsection{Kompilatorer}
När du skriver ett program i C++ skriver du läsbar text. När datorn sedan ska köra din kod måste denna först göras om till något datorn kan läsa. Detta är \emph{kompilatorns} jobb.

Då det är väldigt jobbigt att få C++ att fungera på ett normalt sätt i Windows rekommenderar och antar vi att du kör någon Linux-variant. Exempelvis är Ubuntu (\texttt{http://www.ubuntu.com/}) lätt att installera. Det kan till och med installeras direkt från Windows mha något som kallas Wubi.

\texttt{g++} är den kompilator som oftast används inom tävlingssammanhang, och den som antas när kod kompileras i detta häfte - mer specfikt version 4.6.3.

För att kompilera en fil använder du kommandot

\texttt{\$ g++ -Wall -O2 <filnamn>}

i en terminal, eller kommandoprompt. En fil med namnet \texttt{a.out} kommer då att skapas - detta är filen som du ska köra med hjälp av

\texttt{\$ ./a.out}

\subsection{Hello World!}
Vi inleder med ett kort kodexempel i C++, som inte gör så mycket annat än att skriva ut texten "\texttt{Hello World!}".

\lstinputlisting{cpp/helloworld.cpp}

Börja med att skriva av koden tecken för tecken, spara den som t.ex. \texttt{helloworld.cpp}, kompilera den och kör.

Vi ska nu gå igenom koden rad för rad.

Den första raden, \texttt{\#include} säger att en annan källkodsfils ska inkluderas i vår fil - i detta fall \texttt{iostream}, som hanterar in- och utdata i C++.

Majoriteten av alla inbyggda funktioner i C++ är deklarerade inom vad som kallas en namnrymd (ungefär som ett gatunamn inom programmering) - i detta fall \texttt{std}. För att slippa behöva specifiera namnrymden så fort vi använder någonting ur standardbiblioteket kan vi istället säga att hela filen ska använda det med hjälp av rad 2. Denna rad används i princip alltid.

Rad 4 är början på main-funktionen. Det är i main-funktionen alla C++-program börjar köras när man anropar programmet från en terminal.

Ibland vill man inkludera vanlig text i sina kodfiler för att dokumentera vad det är man gör. Då kan man använda \emph{kommentarer}. Det finns olika sorters kommentarer, och den vanligaste är den som inleds med \texttt{//} på varje rad.

\texttt{cout << "Hello World!";} är en sk sats. Satser är helt enkelt uttryck som faktiskt \emph{gör} saker - i detta fall skriver ut en textsträng. Alla satser måste sluta med ett semikolon i C++.

För att avsluta exekveringen av en funktion kan man använda sig av \texttt{return}-nyckelordet. I main-funktionen innebär detta också att programmet slutar köras. Att vi i detta exempel returnerar just 0 innebär att status-koden för programmet är 0. Detta brukar betyda att programmet exekverade utan något fel.

\subsection{Variabler och datatyper}
Datorns minne är egentligen bara en lång lista med ettor och nollor. Eftersom det dock är svårt att använda minnet i denna form låter C++ och operativsystemet oss att göra saker lite lättare. Det främsta exemplet är att vi istället för att referera till var i listan någonting finns, istället kan använda oss utav ett namn med bokstäver och låta datorn klura ut vilken plats i minnet vi menar - en variabel.

Variabler liknar matematikens variabler, med vissa undantag. Främst behöver vi i C++ veta vad för sorts information vi sparar i en variabel. Är det ett heltal, ett decimaltal, en textsträng, eller något helt annat? Antag exempelvis att vi vill säga åt datorn att variablen $x$ är lika med talet 5.

Detta gör vi med satsen:
\begin{lstlisting}
int x = 5;
\end{lstlisting}

Vi säger här två saker till datorn; först och främst att vi vill ha en variabel $x$ som är ett heltal (int står för engelskans integer, som betyder just heltal) och sedan att $x$ just nu ska ha värdet 5. När vi väl sagt vilken typ $x$ är varken behöver eller får vi säga det igen. En variabel kan alltså aldrig ändra typ efter att den deklarerats.

Till skillnad från matematiska variabler kan vi dock ändra C++-variabler lite hur vi vill. Kanske ändrar vi oss och vill att $x$ ska vara lika med 42 istället. Då gör vi på föjande vis:

\begin{lstlisting}
int x = 5; //x är nu lika med 5
x = 42; //Nu är x lika med 42! Observera avsaknaden av "int" i början
\end{lstlisting}

Kanske vill vi till och med skapa en ny variabel $youJustLostTheGame$ som ska vara lika med $x+2$.

Såhär skulle det se ut:

\begin{lstlisting}
int x = 5;
x = 42;
int youJustLostTheGame = x+2; // = 44
//Om vi nu ändrar x, är youJustLostTheGame fortfarande
//44 - det ändras inte tillsammans med x.
x = 5;
\end{lstlisting}

Det är viktigt här att skilja på hur matematiska variabler skiljer sig från C++. Trots att vi precis sagt att $youJustLostTheGame$ är lika med $x+2$, kommer variabeln \emph{inte} ändras när vi i efterföljande sats sätter $x = 5$. Tilldelningsmodellen skiljer sig alltså från hur algebra fungerar, vilket ofta är förvirrande för nya programmerare.

Variabelnamn får innehålla bokstäver, siffror eller understreck, förutsatt att det inte börjar på en siffra samt inte är någon av de nyckelord som finns i C++.

Det finns många fler exempel på de typer en variabel kan ha. Här visar vi exmpel på några deklarationer:

\begin{lstlisting}
//En char kan lagra ett tecken. Tecken skrivs omgivna av ''.
char ettA = 'a';

//En int kan bara spara tal i intervallet -2147483648 ( = -2^31)
//till 2147483647 ( = 2^31-1)
//En long long kan däremot spara tal mellan 
//ungefär -10^18 (-2^63) till 10^18 (2^63-1)
long long stortTal = 10000000000000LL;

//Värden som bara kan vara sanna eller falska är också vanliga.
//Dessa kallas för booleans och kan ha antingen värdet true eller false.
//Dock fuskar C++ och betraktar allt som inte är 0 som true
//och 0 som false.
bool isJohanAwesome = true;

//Tal med decimaler sparas i doubles. Dessa kan hålla tal
//i intervallet +/- 1.7^(+/- 308). Dock är de inte exakta, utan bara
//en uppskattning (c:a 16 siffrors noggrannhet)
double speedOfAnUnladenSwallow = 15.6132;

//Text-strängar sparar C++ med typen "string" som är en del
//av det inbyggda standardbiblioteket
string herp = "derp";
\end{lstlisting}

\subsection{Input och output}
Inläsning och utskrift av data till terminalen gör man med hjälp av \texttt{cin} och \texttt{cout} som vi tidigare såg. Syntaxen är väldigt enkel:

\lstinputlisting{cpp/input.cpp}

Skriv av koden, kompilera den och kör. Vad gör programmet?

\texttt{cin} och \texttt{cout} kan användas med alla inbyggda numeriska typer, dvs \texttt{int}, \texttt{long long}, \texttt{double}, \texttt{char} osv, samt \texttt{string}.

\subsection{Operatorer}
Nu har vi våra datatyper och variabler. Så, hur gör vi någonting vettigt med dem? Vi har redan sett exempel på två \emph{operatorer} - additionsoperatorn \texttt{+} och tilldelningsoperatorn \texttt{=}.

Det finns ett antal aritmetiska operatorer i C++ som är väldigt bra att kunna:

\lstinputlisting{cpp/operatorer.cpp}

Vid division med noll (antingen genom / eller \%), kommer ett fel uppstå och ditt program avslutas.

Utöver dessa har vi en mängd operatorer som används för att skapa booleska uttryck:

\lstinputlisting{cpp/operatorer2.cpp}

Vi kan också kombinera tilldelningsoperatorn med de flesta andra operatorerna. Om vi vill addera $2$ till en variabel $x$ behöver vi inte skriva \texttt{x = x + 2}, utan kan istället använda operatorn \texttt{+=}. Det betyder alltså samma sak att skriva \texttt{x += 2} som att skriva \texttt{x = x + 2}. Det är inget speciellt med dessa operatorer, utan de gör bara din kod kortare och tydligare. För de två operationerna ``öka ett tal med ett'' och ``minska ett tal med ett'' finns också förkortningarna \texttt{x++} respektive \texttt{x--}.

Prova att leka med operatorerna och se hur de beter sig, speciellt för olika datatyper och kombinationer (hur fungerar division mellan ett heltal och ett flyttal? Addition mellan en boolean och en boolean?).


\section{Kontrollflöde}
\subsection{Villkorssatser}
Än så länge har alla våra program följt en väldigt linjär struktur utan några hopp i källkoden. Det visar sig dock nödvändigt att kunna exekvera vissa satser beroende på om ett booleskt uttryck är sant eller inte.

Antag exempelvis att vi har ett tal $a$ och vill räkna ut absolutbeloppet av det. Absolutbeloppet av ett tal x defineras som bekant såhär:

$$
|x| = 
\left\{
    \begin{array}{ll}
        x & \mbox{om } x \geq 0 \\
        -x & \mbox{annars}
    \end{array}
\right.
$$

Lyckligtvis är detta en barnlek att göra i C++. Ordet ``om'' i den matematiska definitionen ersätts av nyckelordet ``if'' i C++, och nyckelordet ``else'' fungerar som ``annars''.

\lstinputlisting{cpp/villkor.cpp}

Om vi istället har en mer komplex funktion, som:

$$
f(x) = 
\left\{
    \begin{array}{lll}
        1 & \mbox{om } x = 1\\
        \frac{x}{2} & \mbox{om } x \mbox{ jämnt}\\
        3x+1 & \mbox{annars}
    \end{array}
\right.
$$

Vi kan då använda en "else-if-sats". Om den första if-satsen inte var sann, kommer vi att kolla om varje else-if-sats i ordning är sann eller inte (om sådan finns). Annars kommer else-satsen exekveras (återigen, om sådan finns). Exempelvis skulle koden kunna se ut på följande vis:


\lstinputlisting{cpp/villkor2.cpp}

Innan vi går vidare tittar vi på ett sista exempel med villkor.

\lstinputlisting{cpp/villkor3.cpp}


\subsection{Loopar}
Prova att skriva ett program som läser in och summerar 100 heltal.

Och gör det nu utan att skriva 200 rader kod.

Uppenbarligen har vi inte verktygen för att göra det på ett vettigt sätt än, men det ska vi få nu. Låt mig introducera dig till \emph{loopar}:

\lstinputlisting{cpp/loopar.cpp}

\texttt{for} berättar att vi vill upprepa något; precis som med if-satsen, det vi har inom måsvingar. 

Loopen använder sig av tre olika delar, separerade med semikolon, för att kontrollera upprepningen. Den första delen kommer exekveras en gång innan loopen börjar. Mitten-uttrycket måste alltid vara ett booleskt uttryck. Inför varje iteration kommer loopen kontrollera om uttrycket blir \texttt{true} - i så fall körs iterationen. Om uttrycket istället råkar vara false avbryts loopen. Den slutgiltiga satsen körs efter varje iteration.

Om vi ska tolka koden på ren svenska har vi alltså en variabel vid namn $index = 0$. Så länge variabeln är under 100, så läser vi in ett nytt heltal och adderar till vår summa. Därefter ökar vi $index$ med 1. Vi ser alltså att loopen kommer exekveras hela 100 gånger - precis vad vi strävade efter!

Loopar är väldigt användbara. Till exempel kanske vi vill skriva ut alla positiva multiplar av ett visst tal $n$ (dvs $n, 2n, 3n...$) mellan två tal $a$ och $b$. Vi vill alltså ha en variabel $index = $ \emph{första multipeln större än $a$} som initialisering. Inför varje iteration vill vi kontrollera att $index \le b$. Efter varje iteration vill vi hoppa till nästa multipel genom att öka index med $n$.

\lstinputlisting{cpp/loopar2.cpp}

Vi behöver heller inte specifiera någon av satserna:

\lstinputlisting{cpp/loopar3.cpp}

Fundera på vad varje slinga kommer skriva ut innan du kör programmet!

Det är också värt att nämna, att det i C++ ett annat sätt att skriva \texttt{for(;<villkor>;) \{ ... \} }, nämligen \texttt{while(<villkor>) \{ ... \}}. Effekten blir densamma, men det senare är möjligen enklare att läsa.
\subsection{Funktioner}
Vi har redan sett exempel på en funktion - \emph{main} - och ska nu gå in djupare på vad de är och varför de är användbara.

Under avsnittet om villkorsuttryck såg vi hur man kunde beräkna absolutbeloppet av ett heltal med kod. Dock är det en aning omständligt att behöva skriva en if-sats varje gång man vill beräkna beloppet, och det är inte heller särskilt läsbart. Istället skulle man gärna vilja ha ett sätt att ta den operationen och ge den ett symboliskt namn, t ex $abs$. För att utföra operationen på ett tal $x$ skulle man sedan skriva $abs(x)$. Detta är precis vad funktioner gör:


\lstinputlisting{cpp/funktioner.cpp}

För att kunna använda en funktion i C++ så måste den vara definerad tidigare i koden än den används. Det är bland annat därför man har \#include kommandona högst upp i koden. Eftersom det de är till stor del att definera funktioner så måste de vara definerade innan resten av koden.

Funktioner behöver inte returnera något heller. Det är ofta man vill ha en funktion som endast gör något men inte nödvändigtvis ger tillbaka ett svar. För att illustrera detta så behöver vi också nämna att variabler, som vi hittils endast har deklarerat innuti funktioner också kan vara globala. Dvs de är inte innuti någon funktion utan är tillgängliga från alla funktioner. Då en funktion inte ska returnera något så skriver man returntypen ``void" (iställer för ``int" som vi har använt hittils)

\begin{lstlisting}
#include <iostream>

using namespace std;

int counter;
int addToCounter (int num) {
	counter += num;
	cout << "Summan av alla tal är nu: " << counter << endl;
}

int main () {
	while (true) {
		int tal;
		cout << "Skriv ett tal: "<< endl;
		cin >> tal;
		addToCounter (tal);
	}
}
\end{lstlisting}

\section{Datastrukturer}
\subsection{Arrayer}
För att spara många värden, t ex 1000 heltal, skulle vi just nu behöva deklarera 1000 variabler i vår kod. Vill vi sedan läsa in 1000 heltal skulle vi behöva kopiera vår kod 1000 gånger. Detta är självklart inte rimligt eftersom det är så pass vanligt att göra i programmering.  Exempelvis har datorskärmar idag har flera miljoner pixlar, och att deklarera en variabel för varje är inte genomförbart, speciellt då olika sorters skärmar har olika antal pixlar. 

Istället finns \emph{arrayer}. En array är ungefär en lång rad av variabler som vi kan använda utan att behöva deklarera dem för sig. Syntaxet skiljer sig inte särskilt mycket från vanliga variabler:

\lstinputlisting{cpp/arrayer.cpp}

Indexen som används med arrayer är noll-baserade. Detta betyder att i en array med fem element är \texttt{array[0]} det första elementet och \texttt{array[4]} det sista.

Arrayer behöver inte ha fasta storlekar heller, som vi ser i följande exempel:

\lstinputlisting{cpp/arrayer2.cpp}

När man deklarerar arrayer kan man också samtidigt ge dem värden. Syntaxet för det ser ut såhär:

\lstinputlisting{cpp/arrayer3.cpp}

\subsection{\texttt{struct}}
\subsection{Bitset}

\section{Standard Template Library}
\subsection{\texttt{vector}}
\subsection{Iteratorer}
\subsection{\texttt{string}}
\subsection{\texttt{map}}
\subsection{\texttt{queue}}
\subsection{\texttt{set}}
\subsection{\texttt{sort}}
\subsection{\texttt{next\_permutation}}
\subsection{\texttt{iomanip}}

% End content

\end{document}
