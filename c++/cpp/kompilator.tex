När du skriver ett program i C++ skriver du text som människor kan läsa. När datorn sedan ska köra din kod måste denna först göras om till något datorn kan läsa. Detta är \emph{kompilatorns} jobb.

Vi ger lite korta instruktioner för att komma igång med C++ för OS X och Linux. Windows-användare kan installera t.ex Ubuntu (\texttt{http://www.ubuntu.com/}).

\subsubsection{Linux}
I ett Linux-baserat operativsystem är det nog absolut lättast att komma igång med C++. Installera \texttt{g++} med din packethanterare. I Ubuntu kan du använda kommandot

\texttt{\$ sudo apt-get install g++}

för att installera g++.

För att kompilera en fil använder du kommandot

\texttt{\$ g++ -Wall -O2 -g <filnamn>}

i en terminal (i Ubuntu öppnar man en sådan med genvägen Ctrl+Alt+T. En fil med namnet \texttt{a.out} kommer då att skapas - detta är filen som du ska köra med hjälp av

\texttt{\$ ./a.out}

För att ge indata till programmet skriver du den helt enkelt i terminalen efter att du startat programmet.

Du kan också ge en viss fil som indata till programmet med hjälp av

\texttt{\$ ./a.out < filnamn}

vilket kan vara bekvämt istället för att varje gång skriva i testdatan manuellt.

\subsubsection{OS X}

Det finns en del sätt att installera den vanligaste kompilatorn, g++, på OS X. Det vanligaste är att man installerar XCode, vilket är Apples utvecklingsmiljö för OS X. Detta har fördelen att du, förutom en kompilator, också får en relativt bra editor (även om man kan ha åsikter om denna) för att skriva kod i. Alternativet är att installera en package manager och sedan installera g++ direkt, detta har fördelen att du kan få en nyare version av kompilatorn. Troligtvis är detta dock inte något som kommer att vara nödvändigt för denna kurs.

\paragraph{XCode}
För att installera XCode, gå till \url{https://developer.apple.com/xcode/} och ladda ner XCode. Att installera XCode kommer troligtvis att göras via App Store. Det kan ta ett tag att installera, då nedladdningen är på ett antal gigabyte.

För att programmera i XCode kan du antingen skapa ett projekt i XCode och utveckla där, eller koda i en annan textredigerare.
Att koda i XCode kan vara både enklare och svårare. XCode är ett stort program som tar lång tid att lära sig, men förhoppningsvis så kan man lära sig att programmera enklare program snabbt.

Om du vill programmera i XCode så öppnar du XCode, går du till File (i menyraden) $\rightarrow$ New $\rightarrow$ Project $\rightarrow$ Välj OSX, Application, Command Line Tool $\rightarrow$ Skriv in namnet på ditt projekt $\rightarrow$ Välj var projektet ska sparas. När du har skapat ditt projekt så kan du testa att trycka på knappen Run. Du borde se texten \texttt{"Hello World"} visas i ett outputfönster. Om allt fungerade kan du börja redigera koden i filen \texttt{main.cpp}, som du kan se i en lista till vänster i fönstret.

Det går dock även i OS X att kompilera direkt i terminalen på samma sätt som Linux. Detta är viktigt att lära sig eftersom Linux-baserade operativsystem är de vanligaste på tävlingar där man inte får använda sin egen dator (det har hänt att tävlanden inte vetat hur de ska kompilera sina program när de tävlar för första gången i Linux).

För att göra detta, öppna ett terminalfönster (Program $\rightarrow$ Verktygsprogram $\rightarrow$ Terminal), skriv $ls$ för att visa alla filer i mappen du är i, du kan gå runt mellan olika mappar genom att skriva $cd \;mappnamn$ för att gå in i en mapp, samt $cd \;..$ för att gå upp en mapp. Du kompilerar och kör programmet med samma kommandon som i Linux.
