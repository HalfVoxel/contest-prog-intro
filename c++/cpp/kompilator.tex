När du skriver ett program i C++ skriver du läsbar text. När datorn sedan ska köra din kod måste denna först göras om till något datorn kan läsa. Detta är \emph{kompilatorns} jobb.

Vi ger lite korta instruktioner för att komma igång med C++ för OS X och Linux. Windows-användare kan installera t ex Ubuntu (\texttt{http://www.ubuntu.com/}).

\subsubsection{Linux}
I ett Linux-baserat operativsystem är det nog absolut lättast att komma igång med C++. Installera \texttt{g++} med din packethanterare. I Ubuntu kan du använda kommandot

\texttt{\$ sudo apt-get install g++}

för att installera g++.

För att kompilera en fil använder du kommandot

\texttt{\$ g++ -Wall -O2 -g <filnamn>}

i en terminal eller kommandoprompt - det som i Windows kallas \texttt{cmd.exe}. En fil med namnet \texttt{a.out} eller \texttt{a.exe} kommer då att skapas - detta är filen som du ska köra med hjälp av

\texttt{\$ ./a.out}


\subsubsection{OS X}

\todo[inline]{Aron: Instruktioner för OS X}