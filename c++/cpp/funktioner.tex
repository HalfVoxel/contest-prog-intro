Vi har redan sett exempel på en funktion - \emph{main} - och ska nu gå in djupare på vad de är och varför de är användbara.

Under avsnittet om villkorsuttryck såg vi hur man kunde beräkna absolutbeloppet av ett heltal med kod. Dock är det en aning omständligt att behöva skriva en if-sats varje gång man vill beräkna beloppet, och det är inte heller särskilt läsbart. Istället skulle man gärna vilja ha ett sätt att ta den operationen och ge den ett symboliskt namn, t ex $abs$. För att utföra operationen på ett tal $x$ skulle man sedan skriva $abs(x)$. Detta är precis vad funktioner gör:


\lstinputlisting{cpp/funktioner.cpp}

För att kunna använda en funktion i C++ så måste den vara definerad tidigare i koden än den används. Det är bland annat därför man har \#include-kommandona i början på koden; funktionerna de definerar måste ju deklareras innan de används.

Funktioner behöver inte nödvändigtvis returnera ett värde som i exmeplet. Ofta vill man att en funktion ska utföra en handling men förväntar sig inget "svar". För att illustrera detta så behöver vi också nämna att variabler, som vi hittills endast har deklarerat inuti funktioner också kan vara globala. Detta betyder att de är inte är bundna till ett specifikt funktionsanrop, utan alla funktioner kan läsa värdet på variabeln när som helst. Variabelns tillstånd sparas då mellan olika funktionsanrop. Om man inte vill att en funktion ska returnera ett värde använder man returnttypen ``void'' istället för ``int''.

\lstinputlisting{cpp/funktioner2.cpp}

Funktioner kan också ta flera argument, eller inga argument alls (som main-funktionen). Vi kan till exempel göra en funktion som räknar ut arean av en triangel givet sidlängderna på den.

\lstinputlisting{cpp/funktioner3.cpp}

Funktioner behöver inte bara anropa andra funktioner, utan kan också anropa sig själva. Detta kallas \emph{rekursion}, och är extremt välanvänt inom tävlingsprogrammering.   
Här definerar vi en enkel, matematisk rekursiv funktion:
$$
f(x) = 
\left\{
    \begin{array}{ll}
        x & \mbox{om } x \leq 5 \\
        f (x/2) & \mbox{om } x \mbox{ är jämnt} \\
        f ((x-1)*2) & \mbox{annars}
    \end{array}
\right.
$$

Vi ska nu skriva kod som implementerar denna funktion:

\lstinputlisting{cpp/funktioner4.cpp}

Något som är värt att tänka på i C++ är hur variabler funkar. Modellen C++ använder består av \emph{värden} och \emph{pekare}. Dessa går att se på ungefär som ditt hus och din address. När man tilldelar en variabel ett värde (antingen genom tilldelningsoperatorn eller genom att skicka ett argument) kopierar man det värdet. Man tar helt enkelt huset och bygger upp en exakt kopia av det. Detta är helt rimligt när man har väldigt små hus, t ex ett tält. Att bygga upp en ny skyskrapa är aningen jobbigare, och det är då enklare att bara kopiera addressen till huset istället.

Detta problem uppkommer när inom programmering när man har stora datastrukturer, t ex en sekvens av $100 000$ heltal. Att kopiera hela listen tar alldeles för mycket tid! Vi kan dock säga åt en funktion att ett argument inte ska kopieras, utan peka på samma värde som det ursprungliga; helt enkelt skicka addressen till funktionen istället för att bygga ett nytt hus. Detta gör vi genom att skriva ett \&-tecken före variabelnamnet. 

\lstinputlisting{cpp/funktioner5.cpp}

Om du kör programmet så ser du att den första funktionen inte kommer att ändra på orginalvariablen, medans den andra funktionen kommer att göra det.
