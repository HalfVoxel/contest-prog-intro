\begin{exercise}
Skriv ett program med en rekursiv funktion som tar emot ett heltal $n$ och skriver ut talen $n, n-1, n-2, ..., 1$.
\end{exercise}
\begin{exercise}
Skriv ett program med en rekursiv funktion som tar emot ett heltal $n$ och skriver ut talen $1, 2, 3, ..., n$.
\end{exercise}
\begin{exercise}
Skriv ett program som skriver ut gångertabellen för talen 1 till 10.
\end{exercise}
\begin{exercise}
I det här kapitlet betraktade vi följande funktion:
$$
f(x) =
\left\{
    \begin{array}{lll}
        1 & \mbox{om } x = 1\\
        \frac{x}{2} & \mbox{om } x \mbox{ jämnt}\\
        3x+1 & \mbox{annars}
    \end{array}
\right.
$$

En känd förmodan säger att om man börjar med ett positivt heltal $n$, så kommer följden $f(n)$, $f(f(n))$, $f(f(f(n)))$ och så vidare till slut nå 1. Skriv två program som, för ett visst tal $n$, beräknar hur många applikationer av funktionen som krävs innan man når 1: ett rekursivt, och ett iterativt.
\end{exercise}
\begin{exercise}
Skriv en rekursiv funktion för att räkna ut det $n$:te Fibonaccitalet, $F_n$. Skriv också ett program som gör samma sak, fast iterativt. Vad är skillnaden mellan programmen? Vilket kommer vara snabbast?
\end{exercise}
\begin{exercise}
Skriv ett program som tar emot ett positivt heltal $n$ och avgör för vilka positiva heltal $p$ som $\sqrt[p]{n}$ är ett heltal.
\end{exercise}
\begin{exercise}
Skriv ett program som tar emot ett positivt heltal $n$ och avgör om talet är ett primtal eller inte.
\end{exercise}
\begin{exercise}
$e^x$ kan beräknas med hjälp av serien $e^x = \frac{x^0}{0!} + \frac{x^1}{1!} + \frac{x^2}{2!} + \frac{x^3}{3!} + ...$. Skriv ett program som med hjälp av serien beräknar ett närmevärde till $e^x$. I \texttt{cmath} finns också funktionen \texttt{exp} som du kan jämföra ditt närmevärde med.
\end{exercise}