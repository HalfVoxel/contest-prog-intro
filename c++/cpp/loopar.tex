Prova att skriva ett program som läser in och summerar 100 heltal.

Och gör det nu utan att skriva 200 rader kod.

Uppenbarligen har vi inte verktygen för att göra det på ett vettigt sätt än, men det ska vi få nu. Låt mig introducera dig till \emph{loopar}:

\lstinputlisting{cpp/loopar.cpp}

\texttt{for} berättar att vi vill upprepa något; precis som med if-satsen, det vi har inom måsvingar. 

Loopen använder sig av tre olika delar, separerade med semikolon, för att kontrollera upprepningen. Den första delen kommer exekveras en gång innan loopen börjar. Mitten-uttrycket måste alltid vara ett booleskt uttryck. Inför varje iteration kommer loopen kontrollera om uttrycket blir \texttt{true} - i så fall körs iterationen. Om uttrycket istället råkar vara false avbryts loopen. Den slutgiltiga satsen körs efter varje iteration.

Om vi ska tolka koden på ren svenska har vi alltså en variabel vid namn $index = 0$. Så länge variabeln är under 100, så läser vi in ett nytt heltal och adderar till vår summa. Därefter ökar vi $index$ med 1. Vi ser alltså att loopen kommer exekveras hela 100 gånger - precis vad vi strävade efter!

Loopar är väldigt användbara. Till exempel kanske vi vill skriva ut alla positiva multiplar av ett visst tal $n$ (dvs $n, 2n, 3n...$) mellan två tal $a$ och $b$. Vi vill alltså ha en variabel $index = $ \emph{första multipeln större än $a$} som initialisering. Inför varje iteration vill vi kontrollera att $index \le b$. Efter varje iteration vill vi hoppa till nästa multipel genom att öka index med $n$.

\lstinputlisting{cpp/loopar2.cpp}

Vi behöver heller inte specifiera någon av satserna:

\lstinputlisting{cpp/loopar3.cpp}

Fundera på vad varje loop kommer skriva ut innan du kör programmet!

Det är också värt att nämna att det i C++ finns ett annat sätt att skriva \texttt{for(;<villkor>;) \{ ... \}}, nämligen \texttt{while(<villkor>) \{ ... \}}. Effekten blir densamma, men det senare är möjligen enklare att läsa.
