\begin{exercise}
Skriv ett program som läser in två heltal; längderna av de två katetrarna i en rätvinklig triangel och skriver ut triangelns area.
\end{exercise}
\begin{exercise}
Skriv ett program som läser in två heltal i variablerna \texttt{a} och \texttt{b} och sedan byter plats på dem.
\end{exercise}
\begin{exercise}
Skriv ett program som läser in två heltal i variablerna \texttt{a} och \texttt{b} och sedan byter plats på dem \emph{utan att använda en tredje variabel}.
\end{exercise}
\begin{exercise}
Skriv ett program som läser in två par av flyttal $x_1, y_1, x_2, y_2$. Skriv ut avståndet mellan punkterna $(x_1, y_1)$ och $(x_2, y_2)$.
\end{exercise}
\begin{exercise}
Undersök vad som händer när man blandar olika datatyper. Vad händer om du tilldelar ett heltal ett flyttalsvärde? Ett booleskt värde? Och vice versa.
\end{exercise}
\begin{exercise}
Skriv ett program som läser in två heltal, och skriver ut det största av dem. Tips: en \texttt{bool} med värdet \texttt{true} kan också användas som talet 1, och värdet \texttt{false} som värdet 0.
\end{exercise}