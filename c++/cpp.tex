\documentclass[10pt,a4paper]{article}

% Package declarations

\usepackage[utf8]{inputenc}
\usepackage[swedish]{babel}
\usepackage{amsmath}
\usepackage{amsfonts}
\usepackage{amssymb}
\usepackage{graphicx}
\usepackage{parskip}
\usepackage[left=2cm,right=2cm,top=2cm,bottom=2cm]{geometry}
\usepackage[bookmarks]{hyperref}
\usepackage{listings}
\usepackage[usenames,dvipsnames]{xcolor}

\colorlet{KeywordColor}{RoyalBlue}
\colorlet{CommentColor}{ForestGreen}
\colorlet{StringColor}{Plum}

\definecolor{StringColor}{HTML}{C45C05}
\definecolor{CommentColor}{HTML}{5C980A}
\definecolor{KeywordColor}{HTML}{241488}

% End package declarations

% Listings configuration
\lstset{%
language=C++,
showstringspaces=false,
numbers=left,
tabsize=2,
basicstyle=\small\sffamily,
numberstyle=\tiny,
frame=tb,
columns=fullflexible,
showstringspaces=false,
breaklines=true,
breakatwhitespace=true,
keywordstyle=\color{KeywordColor},
commentstyle=\color{CommentColor}\emph,
stringstyle=\color{StringColor}\texttt
}

\lstset{literate=%
{Ö}{{\"O}}1
{Ä}{{\"A}}1
{Ü}{{\"U}}1
{ß}{{\ss}}2
{ü}{{\"u}}1
{ä}{{\"a}}1
{ö}{{\"o}}1
{å}{{\aa}}1
{Å}{{\AA}}2
}

% End listings configuration

% Begin title, ToC

\begin{document}

\Large{INTRODUKTION TILL C++}
\\
\small{Introduktion till C++}
% End title, ToC

% Begin content

\section{Grunder}
\input{cpp/intro.tex}	
\subsection{Kompilatorer}
När du skriver ett program i C++ skriver du läsbar text. När datorn sedan ska köra din kod måste denna först göras om till något datorn kan läsa. Detta är \emph{kompilatorns} jobb.

Vi ger lite korta instruktioner för att komma igång med C++ för OS X och Linux. Windows-användare kan installera t.ex Ubuntu (\texttt{http://www.ubuntu.com/}).

\subsubsection{Linux}
I ett Linux-baserat operativsystem är det nog absolut lättast att komma igång med C++. Installera \texttt{g++} med din packethanterare. I Ubuntu kan du använda kommandot

\texttt{\$ sudo apt-get install g++}

för att installera g++.

För att kompilera en fil använder du kommandot

\texttt{\$ g++ -Wall -O2 -g <filnamn>}

i en terminal eller kommandoprompt - det som i Windows kallas \texttt{cmd.exe}. En fil med namnet \texttt{a.out} eller \texttt{a.exe} kommer då att skapas - detta är filen som du ska köra med hjälp av

\texttt{\$ ./a.out}


\subsubsection{OS X}

Det finns en del sätt att installera den vanligaste kompilatorn, g++, på OS X. Det vanligaste är att man installerar XCode, vilket är Apples utvecklingsmiljö för OS X. Detta har fördelen att du, förutom en kompilator, också får en relativt bra editor (även om man kan ha åsikter om denna) för att skriva kod i. Alternativet är att installera en package manager och sedan installera g++ direkt, detta har fördelen att du kan få en nyare version av kompilatorn. Troligtvis är detta dock inte något som kommer att vara nödvändigt för denna kurs.

\paragraph{XCode}
För att installera XCode, gå till \url{https://developer.apple.com/xcode/} och ladda ner XCode därifrån. Att installera XCode kommer troligtvis att göras via App Store. Det kan ta ett tag att installera, då nedladdningen är på ett antal gigabyte.

När detta är gjort så kan du antingen skapa ett projekt i XCode och utveckla kod där, eller så kan du börja koda i godtycklig texteditor.
Att koda i XCode kan vara både enklare och svårare. XCode är ett stort program och det tar en lång tid att lära sig, men förhoppningsvis så kan man lära sig att programmera enklare program snabbt.

Kortfattat, om du vill programmera i XCode så öppnar du XCode, går till File (i menyraden) $\rightarrow$ New $\rightarrow$ Project $\rightarrow$ Välj OSX, Application, Command Line Tool $\rightarrow$ Skriv in namnet på ditt projekt $\rightarrow$ Välj var projektet ska sparas. Lyckas du inte med detta så kan du söka på google, det finns utmärkta tutorials på nätet om hur du gör detta. När du har skapat ditt projekt så kan du testa att trycka på knappen Run, du borde se texten Hello World visas i ett outputfönstret. Går allt detta bra så är det bara att börja redigera kod i filen main.cpp som du kan se i en lista till vänster i fönstret.

Det går dock även i OS X att kompilera direkt i terminalen. Detta är värt att lära sig då det är ett sätt som alltid fungerar på tävlingar på nästan precis samma sätt, vilket minskar risken att man misslyckas endast för att man inte visste hur man skulle kompilera programmet.

För att göra detta, öppna ett terminalfönster (Program/Verktygsprogram/Terminal), skriv $ls$ för att visa alla filer i mappen du är i, du kan gå runt mellan olika mappar genom att skriva $cd mappnamn$ för att gå in i en mapp, samt $cd ..$ för att gå upp en mapp.
Antag att du har sparat en fil med källkod, med namnet <filnamn> (t.ex test.cpp), du kompilerar då detta program precis som du skulle göra på Linux (se ovan), dvs:

\texttt{\$ g++ -Wall -O2 -g <filnamn>}

Och kör precis som i Linux också, dvs:

\texttt{\$ ./a.out}
\subsection{Hello World!}
\input{cpp/helloworld.tex}
\subsection{Variabler och datatyper}
\input{cpp/variabler.tex}
\subsection{Input och output}
\input{cpp/input.tex}
\subsection{Operatorer}
\input{cpp/operatorer.tex}
\subsection{Övningar}
\begin{exercise}
Skriv ett program som läser in två heltal; längderna av de två katetrarna i en rätvinklig triangel och skriver ut triangelns area.
\end{exercise}
Skriv ett program som läser in två heltal i variablerna \texttt{a} och \texttt{b} och sedan byter plats på dem.
\begin{exercise}
Skriv ett program som läser in två heltal i variablerna \texttt{a} och \texttt{b} och sedan byter plats på dem \emph{utan att använda en tredje variabel}.
\end{exercise}
Skriv ett program som läser in två par av flyttal $x_1, y_1, x_2, y_2$. Skriv ut avståndet mellan punkterna $(x_1, y_1)$ och $(x_2, y_2)$.
\begin{exercise}
\end{exercise}
\begin{exercise}
\end{exercise}
\begin{exercise}
\end{exercise}
\begin{exercise}
\end{exercise}
\begin{exercise}
\end{exercise}
\begin{exercise}
\end{exercise}
\begin{exercise}
\end{exercise}
\begin{exercise}
\end{exercise}
\begin{exercise}
Skriv ett program som läser in två heltal, och skriver ut det största av dem. Tips: en \texttt{bool} med värdet \texttt{true} kan också användas som talet 1, och värdet \texttt{false} som värdet 0.
\end{exercise}

\section{Kontrollflöde}
\subsection{Villkorssatser}
\input{cpp/villkor.tex}
\subsection{Loopar}
\input{cpp/loopar.tex}
\subsection{Funktioner}
Vi har redan sett exempel på en funktion - \emph{main} - och ska nu gå in djupare på vad de är och varför de är användbara.

Under avsnittet om villkorsuttryck såg vi hur man kunde beräkna absolutbeloppet av ett heltal med kod. Dock är det en aning omständligt att behöva skriva en if-sats varje gång man vill beräkna beloppet, och det är inte heller särskilt läsbart. Istället skulle man gärna vilja ha ett sätt att ta den operationen och ge den ett symboliskt namn, t ex $abs$. För att utföra operationen på ett tal $x$ skulle man sedan skriva $abs(x)$. Detta är precis vad funktioner gör:


\lstinputlisting{cpp/funktioner.cpp}

För att kunna använda en funktion i C++ så måste den vara definerad tidigare i koden än den används. Det är bland annat därför man har \#include kommandona högst upp i koden. Eftersom det de är till stor del att definera funktioner så måste de vara definerade innan resten av koden.

Funktioner behöver inte returnera något heller. Det är ofta man vill ha en funktion som endast gör något men inte nödvändigtvis ger tillbaka ett svar. För att illustrera detta så behöver vi också nämna att variabler, som vi hittils endast har deklarerat innuti funktioner också kan vara globala. Dvs de är inte innuti någon funktion utan är tillgängliga från alla funktioner. Då en funktion inte ska returnera något så skriver man returntypen ``void" (iställer för ``int" som vi har använt hittils)

\lstinputlisting{cpp/funktioner2.cpp}

Funktioner kan också ta flera argument, eller inga argument alls (som main funktionen). Vi kan till exempel göra en funktion som räknar ut arean av en triangel givet sidlängderna på den (Herons formel ftw! Höjd och bredd är tråkigt).

\lstinputlisting{cpp/funktioner3.cpp}

Funktioner kan som vi har sett anropa andra funktioner, men de kan också andropa sig själva vilket kan vara väldigt användbart.
Här är en enkel rekursiv funktion.
$$
f(x) = 
\left\{
    \begin{array}{ll}
        x & \mbox{om } x \leq 5 \\
        f (x/2) & \mbox{om } x \mbox{ är jämnt} \\
        f ((x-1)*2) & \mbox{annars}
    \end{array}
\right.
$$

Vi ska skriva en funktion som fungerar som ovan och skriver ut varje steg.

\lstinputlisting{cpp/funktioner4.cpp}

I alla ovanstående funktioner har vi skickat argument, men om vi skickar variabeln x så kommer allt som funktionen gör med den inte att påverka orginal variabeln.
Skriver vi ett \& tecken framför argumentnamnet i funktionsdeklarationen så kommer funktionen att skickas som en referens och inte som ett värde.

\lstinputlisting{cpp/funktioner5.cpp}

Om du kör programmet så ser du att den första funktionen inte kommer att ändra på orginalvariablen, medans den andra funktionen kommer att göra det.
\subsection{Övningar}
\begin{exercise}
Skriv ett program med en rekursiv funktion som tar emot ett heltal $n$ och skriver ut talen $n, n-1, n-2, ..., 1$.
\end{exercise}
\begin{exercise}
Skriv ett program som skriver ut gångertabellen för talen 1 till 10.
\end{exercise}
\begin{exercise}
I det här kapitlet betraktade vi följande funktion:
$$
f(x) =
\left\{
    \begin{array}{lll}
        1 & \mbox{om } x = 1\\
        \frac{x}{2} & \mbox{om } x \mbox{ jämnt}\\
        3x+1 & \mbox{annars}
    \end{array}
\right.
$$

En känd förmodan säger att om man börjar med ett positivt heltal $n$, så kommer följden $f(n)$, $f(f(n))$, $f(f(f(n)))$ och så vidare till slut nå 1. Skriv två program som, för ett visst tal $n$, beräknar hur många applikationer av funktionen som krävs innan man når 1: ett rekursivt, och ett iterativt.
\end{exercise}
\begin{exercise}
Skriv en rekursiv funktion för att räkna ut det $n$:te Fibonaccitalet, $F_n$. Skriv också ett program som gör samma sak, fast iterativt. Vad är skillnaden mellan programmen? Vilket kommer vara snabbast?
\end{exercise}
\begin{exercise}
Skriv ett program som tar emot ett positivt heltal $n$ och avgör för vilka positiva heltal $p$ som $\sqrt{n}^p$ är ett heltal.
\end{exercise}
\begin{exercise}
Skriv ett program som tar emot ett positivt heltal $n$ och avgör om talet är ett primtal eller inte.
\end{exercise}
\begin{exercise}
$e^x$ kan beräknas med hjälp av serien $e^x = \frac{x^0}{0!} + \frac{x^1}{1!} + \frac{x^2}{2!} + \frac{x^3}{3!} + ...$. Skriv ett program som med hjälp av serien beräknar ett närmevärde till $e^x$. I \texttt{cmath} finns också funktionen \texttt{exp} som du kan jämföra ditt närmevärde med.
\end{exercise}

\section{Datastrukturer}
\subsection{Arrayer}
För att spara många värden, t ex 1000 heltal, skulle vi just nu behöva deklarera 1000 variabler i vår kod. Vill vi sedan läsa in 1000 heltal skulle vi behöva kopiera vår kod 1000 gånger. Detta är självklart inte rimligt eftersom det är så pass vanligt att göra i programmering.  Exempelvis har datorskärmar idag flera miljoner pixlar, och att deklarera en variabel för varje pixel är inte genomförbart, särskilt då olika sorters skärmar har olika antal pixlar. 

Istället finns \emph{arrayer}. En array är ungefär en lång rad av variabler som vi kan använda utan att behöva deklarera dem för sig. Syntaxet skiljer sig inte särskilt mycket från vanliga variabler:

\lstinputlisting{cpp/arrayer.cpp}

Indexen som används med arrayer är noll-baserade. Detta betyder att i en array med fem element är \texttt{array[0]} det första elementet och \texttt{array[4]} det sista.

Arrayer behöver inte ha fasta storlekar heller, som vi ser i följande exempel:

\lstinputlisting{cpp/arrayer2.cpp}

När man deklarerar arrayer kan man också samtidigt ge dem värden. Syntaxet för det ser ut såhär:

\lstinputlisting{cpp/arrayer3.cpp}

Nyttigt att veta är att strängar också kan betraktas som en array av \texttt{char}. Vi kan alltså skriva till exempel:

\begin{lstlisting}
string palindrome = "Madam Adam";
for(int i = 0; i < 10; i++){
	cout << "Tecknet på index " << i << " = " << palindrome[i] << endl;
}
\end{lstlisting}

Man kan också skapa arrayer som innehåller arrayer, s k \emph{multidimensionella} arrayer. Tillämpningar är exempelvis rutnät. Vi ska nu titta på vårt första lite längre och mer komplicerade kodexempel. Det tar emot ett \emph{Röj}-bräde med minor (*) och tomma rutor (.) och sedan skriver ut hur många minor som finns runt varje ruta. Här utnyttjar vi också det faktum att en sträng kan användas som en array av tecken:

\lstinputlisting{cpp/arrayer4.cpp}

Prova till exempel brädena:

\begin{verbatim}
4 4
...*
..*.
.*.*
*.*.
\end{verbatim}

och

\begin{verbatim}
3 3
***
*.*
***
\end{verbatim}

Arrayer behöver inte bara vara tvådimensionella, utan kan ha godtyckligt många dimensioner
\subsection{\texttt{struct}}
\subsection{Bitset}
\subsection{Övningar}
\input{cpp/structovn.tex}

\section{Standard Template Library}
\subsection{\texttt{vector}}
\subsection{Iteratorer}
\subsection{\texttt{string}}
\subsection{\texttt{map}}
\subsection{\texttt{queue}}
\subsection{\texttt{set}}
\subsection{\texttt{sort}}
\subsection{\texttt{next\_permutation}}
\subsection{\texttt{iomanip}}

% End content

\end{document}
